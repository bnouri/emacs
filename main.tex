\RequirePackage[l2tabu, orthodox]{nag}
\documentclass[10pt,oneside, twocolumn]{article}
\usepackage{_sty/art,_sty/sh,_sty/dom,_sty/pdf,_sty/shx}
%******************************
%       *** Title ***
%******************************
\title{Emacs}
\author{bnouri}
\date{\small \today}
% \def\YY{2023} \def\MM{April} \def\DD{21}
% \date{\small \MM~\DD,~\YY}

%******************************
%   *** Domesti\TT{C-Tunings ***
%******************************
\setlength{\parindent}{0pt}
%\setlength{\parskip}{0.5\baselineskip}
%
%******************************

\input{_sty/meta}
\begin{document}
\bstctlcite{IEEEexample:BSTcontrol} %Keep it right here to add controls in the bib file into effect.
\input{_sty/fm}
%%%%%%%%%%%%%%%%%%%%%%%%%%%%%%%
% *** The Body of Document ***
%%%%%%%%%%%%%%%%%%%%%%%%%%%%%%%
\vfill
\pagebreak
%======================================
\section{Help}
%======================================
Youtube:~\href{https://www.youtube.com/watch?v=jPkIaqSh3cA}{The Basics of Emacs as a Text Editor}
\begin{tabularx}{0.5\textwidth}{lX}
\TT{C-x k} &Stops the tutorial  \\
\end{tabularx}


%======================================
\section{Basics}
%======================================
\begin{tabularx}{0.5\textwidth}{lX}
\TT{C-x C+}& zoom in\\
\TT{C-x C-}& zoom out\\
\TT{C-x C-c} & exit  \\
\TT{C-g} & quit a partially entered command\\
\TT{C-l} & clear screen \& more\\
\end{tabularx}

%======================================
\section{Split}
%======================================
\begin{tabularx}{0.5\textwidth}{lX}
\TT{C-x 0} & exit/close current pane/buffer  \\
\TT{C-x 2} & H-split\\
\TT{C-x 3} & V-split\\
\TT{C-x o} & switch between pane\\
\end{tabularx}

%======================================
\section{File manager}
%======================================
\begin{tabularx}{0.5\textwidth}{lX}
\TT{C-x C-f} & To open file manager in a pane\\
\TT{C-x 4f} & opens file manager in the other pane\\
\TT{C-x C-c} & close the edited file (and decide if save)\\
\TT{C-x C-s} & to save the file, then \TT{C-x C-C} to close\\
\end{tabularx}


%======================================
\section{Search/Replace}
%======================================
\begin{tabularx}{0.5\textwidth}{lX}
    \TT{C-s} & open isearch to search forward\\
    & \TT{C-s} go to next instance\\
    & \TT{C-r} go reverse\\
\TT{C-r} & open isearch to search backward\\
   & \TT{C-r} go previous instance\\
    \TT{M\% "this" "that"} & Replace "this" with "that"\\
    \TT{M-u} & capitalize the word cursor is on its begining / is highlighted\\
    \TT{M-l} & change to all letters of a word to lower case \\
\end{tabularx}

%======================================
\section{Spelling}
%======================================
\begin{tabularx}{0.5\textwidth}{lX}
\TT{M-\$} & Spellcheck the word under cursor\\
\TT{M-u} & capitalize the word cursor is on its begining / is highlighted\\
\TT{M-l} & change to all letters of a word to lower case \\
\end{tabularx}
aspell-en
%======================================
\section{Capital}
%======================================
\begin{tabularx}{0.5\textwidth}{lX}
\TT{M-c} & Cpitalize letter under cursor\\
\TT{M-u} & capitalize the word cursor is on its begining / is highlighted\\
\TT{M-l} & change to all letters of a word to lower case \\
\end{tabularx}

%======================================
\section{Cursor}
%======================================
\begin{tabularx}{0.5\textwidth}{lX}
\TT{C-v} & page-down  \\
\TT{M-v} & page-up\\
\TT{C-p} & previous line $\uparrow$\\
\TT{C-n} & next line $\downarrow$\\
\TT{C-b} & bkward  $\leftarrow$ char\\
\TT{C-f} & forward $\rightarrow$ char\\
\TT{M-b} & bkward  $\leftarrow$ word\\
\TT{M-f} & forward $\rightarrow$ word\\
\TT{C-e} & end of line\\
\TT{C-A} & begining of line\\
\TT{M-e} & end of sentence \\
\TT{M-A} & begining of sentence\\
\TT{M-\} } & Jump forward a para\\
\TT{M-\{ } & Jump up a para\\
\TT{M->} & jump to to end of file \\
\TT{M-<} & Jump to begining of file\\
\TT{M-g g} & Jump to line \\
\TT{Esc nn C-n} & Jump "nn" line \\
\TT{Esc nn C-f} & jump "nn" caharactre\\
\end{tabularx}

%======================================
\section{Cut-Copy-Paste}
%======================================
\begin{tabularx}{0.5\textwidth}{lX}
& Use mouse to select text \\
\TT{space C-F} &  Select char by char \\
\TT{Space C-n} & Select line by line \\
\TT{M-w} & copy  \\
\TT{C-w} & cut \\
\TT{C-K} & Cuts from cursor to end of line\\
\TT{C-x bkspace} & Cuts from cursor to begining of line\\
\TT{C-z char} & Cuts from cursor to next appearance of "char"\\
\TT{C-y} & paste \\
\TT{C-x u} & undo\\
\\
\TT{Esc n d} & deletes n word starting from cursor\\
\TT{C-h v kill-ring} & opens history of cut/paste,...\\
\end{tabularx}
All cuts, deletes, pastes, ... go to a buffer called Kill-ring.\\

%======================================
\section{Buffers}
%======================================
\begin{tabularx}{0.5\textwidth}{lX}
\TT{c-x 0} & close the buffer (pane)\\
\TT{c-x b buffer-name} & Opens the buffer-name, if not exist creats\\
\TT{c-x k buffer-name} & kills/closes buffer-name \\
\TT{c-x c-b} & list currently open buffers\\
\end{tabularx}

%\vfill \pagebreak
%======================================
\section{Packages}
%======================================
\TT{M-x package-list-packages} $\rightarrow$ \TT{C-s theme} to search "theme"\\
\begin{tabularx}{0.5\textwidth}{lX}
\TT{M-x customize-themes} & switch to a buffer named \TT{*Custom Themes*}  \\
\TT{M-x disable-theme} & enter name of the theme\\
\TT{M-x eshell} & Opens e-shell\\
\TT{M-x load-theme}& \texttt{misterioso}\\
\end{tabularx}
dichromacy, light-blue, adwaita, manoj-dark, misterioso, tango, tango-dark, tsdh-dark, tsdh-light, wheatgrass, whiteboard, wombat, deeper-blue
%
\begin{tabularx}{0.5\textwidth}{lX}

\end{tabularx}    

% %******************************
% %      *** Appendix ***
% %******************************
% %\appendix
% %\section{title}\label{sec:100}
% %\section{title}\label{sec:101}
% %******************************
% %     *** References ***
% %******************************
% %\bibliography{bib/Ref}
% %
% %%\input{_xuti/Sample.tex}
\end{document}




